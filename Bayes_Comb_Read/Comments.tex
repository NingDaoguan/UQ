\documentclass{article}

\begin{document}
\begin{center}
\Large Comments, Jonathan Goodman, July, 2014 \normalsize
\end{center}

\noindent {\bf ElasticConstants.pdf:} Miguel A. Aguilo Laura Swiler, Angel Urbina, 
``AN OVERVIEW OF INVERSE MATERIAL IDENTIFICATION WITHIN THE FRAMEWORKS OF DETERMINISTIC AND STOCHASTIC PARAMETER ESTIMATION'', 
International Journal for Uncertainty Quantification, 3 (4): 289�319 (2013)

\begin{enumerate}
\item Specific problem: recover spatially varying elastic modulus from measurements.

\item Two kinds of parameter estimation: deterministic and stochastic.
Stochastic gives uncertainty estimates.
Bayesian: ``... multiple solutions of the unknown may be consistent with the observations.''

\item References for elliptic inverse problems.

\item Likelihood function takes into account {\em ECE}, ``error in the constitutive equation''.

\item Represent the spatially varying elastic modulus using radial basis functions with
unknown centers and length parameters and coefficients.
Ill conditioned in radial basis coefficients $\longrightarrow$ regularize.

\item Kaipio, J. and Somersalo, E., {\em Statistical and Computational Inverse Problems}, Springer, 
NewYork, 2005.

\item Numerics: fake data, 1D, multiplicitive Gaussian observation noise, small, known variance.

\item MCMC, delayed rejection adaptive algorithm http://www.helsinki.fi/?mjlaine/dram/

\item Haario, H., Laine, M., Mira, A., and Saksman, E., DRAM:EfficientadaptiveMCMC, Stat.Comput., 
16(4):339�354, 2006.

\item Estimated auto-correlation function from MCMC data.  
Standard likelihood gives auto-correlation time in the thousands.
The ECE likelihood $\tau$ is only 100 or so.
Run length 20,000 samples, first 5,000 discarded (burn in).

\item 15 observation values, 5 radial basis functions (common length scale, uniformly spaced centers?)
Some of the results seem to accurate to be represented with 5 radial basis functions of any kind.

\end{enumerate}

\noindent {\bf TemponeAbstract.pdf:} Daesang Kim, Fabrizio Bisetti, Quan Long, Raul Tempone, 
and Omar Knio, {\em Spectral Uncertainty Quantification and Optimal Bayesian Experimental Design of Combustion Reaction Systems using Sparse Adaptive Polynomial Chaos Expansion}, conference abstract, 2014

\begin{enumerate}

\item Polynomial chaos 

\item Hydrogen flame parameters, then methane system parameters

\end{enumerate}

\noindent {\bf AIAA.pdf:} Paul G. Constantine, Alireza Doostan, Qiqi Wang, Gianluca Iaccarino, 
{\em A Surrogate Accelerated Bayesian Inverse Analysis of the HyShot II Flight Data}, 
AIAA 2011-2037, 52nd AIAA/ASME/ASCE/AHS/ASC Structures, Structural Dynamics and Materials Conference 19th AIAA 2011-2037 4 - 7 April 2011, Denver, Colorado

\begin{enumerate}

\item Hypersonic RANS viscous flow with combustion, model of a scramjet, 11 pressure measurements,
6 estimated parameters: speed, angle of attack, altitude, top wall temperature, bottom wall temperature,
one turbulence variable, fake data.  

\item Full fluid (RANS) solve for a likelihood call, or a surrogate m.odel.  1800 fluid solves.

\item Silva, D. and Skilling, J., Data Analysis: A Bayesian Tutorial, Oxford University Press, 2006.

\item Compared surrogate models, polynomial chaos, Gaussian process, Beylkin low rank approx -- about
the same.  Crummy numerics.

\end{enumerate}

\noindent {\bf BramanOliverRaman.pdf:} Kalen Braman, Todd A. Oliver, Venkat Raman, 
{\em Bayesian analysis of syngas chemistry models}, 
Combustion Theory and Modelling, 2013
Vol. 17, No. 5, 858�887

\begin{enumerate}

\item Full Bayesian posterior sampling for combustion chemical kinetics parameters.  

\item Lots of references for UQ in chemical reaction fitting.

\item ``It is important to note that the different parameters are correlated, and that varying one 
without adjusting the rest is not meaningful.''

\item T. Russi, A. Packard, R. Feeley, and M. Frenklach, ``Sensitivity analysis of uncertainty in model prediction'', J. Phys. Chem. A 112 (2008), pp. 2579�2588. Available at http://dx.doi.org/10.1021/jp076861c.

\item Adaptive multi-level sampling algorithm, refs [28,29], related to sequential sampling?
QUESO library.

\item Uses our Davis mechanism, and two others.

\item Only pre-exponential factors included in Bayesian parameter sets, not activation energies.

\item Calibrates to flame speed data, 18 measurements in all.

\item Select parameters based on some kind of a-priori sensitivity analysis.

\item Simulations done with CHEMKIN and PREMIX

\item 15,000 processor hours, ... the majority of the computational time is spent computing the 
flame speed.

\end{enumerate}



\end{document}
