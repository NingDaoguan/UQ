\documentclass[11pt]{article}
% RFP specifically says to use 11 point type and 1 inch margins
\usepackage{graphicx}
\usepackage{epsf,color}
\textwidth=6.5in\oddsidemargin=0in \evensidemargin=0in \topmargin
0pt \advance \topmargin by -\headheight \advance \topmargin by
-\headsep \textheight 9.0in

%\textwidth=6.5in\oddsidemargin=0in \evensidemargin=0in \topmargin
%0pt \advance \topmargin by -\headheight \advance \topmargin by
%-\headsep \textheight 8.9in

\usepackage{amsmath}
\usepackage{graphicx}
\usepackage{dcolumn}
\usepackage{multirow}
\usepackage{wrapfig}
\usepackage[compact]{titlesec}
\usepackage{url}                                                               


%\usepackage[plain]{fullpage}
\usepackage{amsfonts}
%\usepackage{lastpage}
%\usepackage{fancyhdr}

\usepackage[version=3]{mhchem} 
% you can use this command to skip chunks of your document
% just put the command around the chunk like this
% \comment{ ...the chunk... }
\newcommand{\comment}[1]{}

%\newcommand{\MarginPar}[1]{\hspace{1sp}\marginpar{\tiny\sffamily\raggedright\hspace{1sp}#1}}
\setlength{\marginparwidth}{0.75in}
\newcommand{\MarginPar}[1]{\marginpar{%
\vskip-\baselineskip %raise the marginpar a bit
\raggedright\tiny\sffamily
\hrule\smallskip{\color{red}#1}\par\smallskip\hrule}}

\definecolor{drkgrn}{rgb}{0.043,0.341,0.2274}
\newcommand{\remrg}[1]{ {\it \color{drkgrn} \{#1 -RG\}}}

%\renewcommand{\baselinestretch}{1.05} % = 1.0 Single space; = 2.0 Double
\renewcommand{\baselinestretch}{1.0} % = 1.0 Single space; = 2.0 Double

%\renewcommand{\refname}{Literature Cited}
%------------------------

%\pagestyle{empty}  % No page numbers
%\textfloatsep 0mm
%\abovecaptionskip 1mm

\begin{document}

%\pagestyle{plain}
%\pagenumbering{roman}

\subsection{Modeling and simulation for energy storage}
A recent DOE Basic Energy Sciences (BES) workshop report states: 
``Revolutionary breakthroughs in Electrical Energy Storage have been singled out as perhaps the most crucial need
for this nation's secure energy future''  (\cite{ees_rpt}, p. xii).
The promise of electric-drive vehicles (EDVs) can only be met if batteries having both high energy density and increased safety 
and reliability are developed.  Current solutions, including exploding laptops and expensive electric vehicles that burst into 
flame and/or can only travel 100 miles before requiring a full night to recharge, are inadequate for widespread adoption.
Lithium ion batteries (LIBs) present a truly multiscale system, with important macroscopic behavior
(e.g., catastrophic failure such as fire) determined by intricate microscale
interactions (e.g., as a result of dendritic growth on lithium particles in
one location deep inside the battery pack). 
%It is extremely difficult, where
%possible at all, to examine the mechanisms involved---especially, to
%elucidate the microscopic origin of macroscopic phenomenon---experimentally.
The highly structured, heterogeneous nature of battery systems implies we must model a coupled multiphysics system, rather
than one system presumed to govern the full device.   
In theory one can enumerate at least 6 levels of detail that are tied together in some way: 
atomistic (quantum), atomistic (MD), particles, electrode, cell, pack.  The problem certainly represents a potential
extreme scale computation, as relevant phenomena over 10 orders of magnitude must be resolved and coupled
in some way.

HERE, SPACE PERMITTING, I WOULD QUICKLY ENUMERATE THE FORMS OF THE EQUATIONS ENCOUNTERED AT THE VARIOUS SCALES:
quantum (Schroedinger), MD (Newton's law), particle (diffusion + Butler-Vollmer to model interface kinetics), 
electrolyte (convection diffusion), electrode (mechanical constitutive equations), cell (diffusion, thermal transport).
  
%``Simulation of lithium-ion battery models requires simultaneous evaluation of concentration and potential fields, %in both solid as well as liquid phases. In addition, the porous nature of the battery electrodes leads to highly %nonlinear and heterogeneous electrochemical reaction kinetics." \cite{Subramanian:2009}

Representative examples of battery models include \cite{ex1}, in which a fully resolved 3D electrode simulated using a combination
of convection-diffusion and Butler-Volmer equations. This model uses approximately 25 parameters derived by the authors from experiment.
Another example is \cite{ex2}, which simulates a 3D electrode, including the constitutive relations that are first steps toward
incorporating stress/strain in the models. This model contains approximately 50 parameters.  Another interesting case is \cite{ex3}, where models at three scales (particle, electrode, cell) are coupled via boundary conditions and forcing terms. This model also contains roughly 50 parameters.

The current paradigm for virtually all battery simulations (including those above)
 is the use of effective parameters derived from single-scale experiments
used as fixed values in single-scale simulations.  However, this is problematic: first, we do not propogate uncertainty in 
these parameters through the model; second, we miss the chance to use information \emph{across scales} to optimally understand
the system.  Our proposed work specifically addresses this shortcoming in the current paradigm by offering the means to 
combine modeling and data \emph{across a hierarchy of experiments}
to reduce uncertainty in the underlying physics parameters and improve predictive capability.

The hope for such a program rests on the fact that hierarchies of experiments exist addressing virtually all aspects 
of the heterogeneous battery system (TO BE NEATENED/FINISHED (E.G. MORE REFS)):\\
\textbf{diffusivities}: (microscale) single-crystal measurements, molecular diffusivity  \cite{Chung:2011}; (mesoscale) in-situ/single particle measurements \cite{Cui:2012};  (macroscale) GITT/PITT (Routinely carried out using composite electrodes) \cite{Wen01121979}\\
%Multi-scale measurements hel p resolve disputes in transport mechanism
\textbf{rate constants}: (microscale) reactive TEM� (PNNL) ; (mesoscale) Aurbach and others;  
(macroscale) electrochemical impedence spectroscopy (only ``effective reactivities" are of interest to the macro-models) \\
%Multi-scale measurements help eliminate multiple �optima�
\textbf{open circuit potentials, entropy measurements, decomposition potentials}:  
(microscale) composition versus lattice structure (e.g., XRD Ceder, Ohzuku);
(mesoscale) ECQM???;
(macroscale) C/50 discharge\\
%�Voltage fade�???, non-stoichiometric �phases�???, hysteresis!!!, decompn. potentials fairly well documented
\textbf{conductivities (electronic, ionic)}:  
(microscale) quantum transport;
(mesoscale) Measure properties of individual components and use mixing rules (effective medium theory);
(macroscale) Measure effective conductivities \\
%Not strictly multiscale; need to think about this more
\textbf{mechanical properties}: 
(microscale) in-situ TEM \cite{Wang:2012};
(mesoscale) single particle and in situ electrode level \cite{Qi:2010,Verbrugge:1999};
(macroscale) pressure transducers  \\
%Rational design across scales

We propose to use the multiscale implicit sampling and other frameworks developed under this proposal to
address the origin and effects of pressure fluctuations in batteries for electric drive vehicles.  Specifically, we intend to
use macroscopic experiments,
% microscopic calculations (quantum),
micro-(particle level) and meso-scale (electrode level) experiments to
understand the small scale---especially mesoscale (because this is where the
battery designer has the most degrees of freedom)---origin of macroscale
pressure trends known to correlate with battery degradation (e.g.,
capacity fade) and safety (e.g., thermal runaway, a.k.a. fire) issues.

\subsection{Formulation of multi-scale implicit sampling for modeling pressure in battery systems}
The fluctuation of pressure with time can critically affect performance, degradation, and safety of batteries.
We label the scales at which to model battery processes
somewhat arbitrarily as ``micro" (roughly, nanometers), ``meso" (microns), and ``macro" (millimeters).  
Referring to the above general formulation of multi-scale implicit sampling,  let the state/parameters
at these scales be denoted $x^0$, $x^1$, and $x^2$, respectively.
  The macroscale state variable $x^2$ of interest is pressure.  We can measure it via data $y^2$ which consists
of detailed signals from transducers.  
%For purposes of this discussion, we can assume the ``observation model" 
%$y^2 = h(x^2) + \textrm{ noise}$ is perfect, so $y^2 \equiv $x^2$   
  Macroscale pressure is in fact the manifestation of many effects at the mesoscale: conversion of electrolyte
to the gas phase; the sponge like nature of the porous electrode; elastic properties of the binder (the binder
is the "glue" that holds the particles in their porous structure); the porosity and toruosity of the separater.  The pressure is the sum of all these effects combined across the whole cell.  Let $x^1$ represent the state,
described at the mesoscale in terms of such quantities as layer thicknesses (substrate, epoxy layer,
current collector, active material, entire electrode), volume fractions (particles, binder, pores), 
porosity, tortuosity, particle sizes and distributions  
(see, e.g., \cite{Sethuraman2012334}) and other parameters governing the makeup and morphology of the heterogeneous
electrode.
Let $y^1$ be data
gathered regarding mesoscale phenomena such as stress, strain, swelling, and other mechanical properties.  
Our goal is to
incorporate all available data, i.e. both $y^1$ (measuring electrode properties: stress, strain, etc.) and $y^2$ (measuring cell properties: pressure), into the estimate of
$$
	p(x^2,x^1 | y^2,y^1) \propto p(y^2|x^2) p(x^2|x^1) p(x^1),
$$
the conditional expection of cell-level pressure and electrode-level stress and strain, given both
cell and electrode level data.
The first term on the right hand side represents our observation model of "pressure".
The second term come from a physical or statistical model of how the various mesocsale quantities 
affect pressure.  The last term is our guess at what the mesoscale parameters really will be.
From this equation we can see how both the data (first term) and physical model (second term) are being
incorporated.  

To sample from this distribution by implicit sampling, we will follow the steps outlined above, specifically:
\begin{itemize} \itemsep -0.2em
\item form $F = -\log( p)$
\item minimize it, get $\sigma_F$, or even (from a population based algorithm) $\{\sigma^i_F\}_{i=1}^{Npop}$
\item sample from it by repeating:\\
 -- sample $\xi$ from Gaussian\\
 -- solve $F-\sigma_F = \xi^T \xi$. \\
\end{itemize}
With these samples in hand we can calculate any desired statistics of the macroscale pressure. 
What have we accomplished?  We have demonstrated a formulation that allows physics-based models (e.g., 
how pressure arises from mesoscale phenomena) to inform our interpretation of macroscale measurements. 
Conversely, we have demonstrated how the physics-based model can ``learn" from the macroscale measurements.
\MarginPar{emphasize that this could be extreme scale science}

Interestingly, we can also reverse the formulation.  \emph{This is perhaps more of interest, in fact, because
macroscale pressure measurements are \textbf{more} reliable than meso- and micro-scale measurements.}  By applying
the formalism from scale $2$ \emph{to} scale $1$, we have 
$$
	p(x^1,x^2 | y^1,y^2) \propto p(y^1|x^1) p(x^1|x^2) p(x^2).
$$
In this case the physics model is used to compute $p(x^1|x^2)$.  Because of the presumed (in this case)
many-to-one relation between mesoscale parameters and macroscopice pressure, this term acts as  constraint (what $x^1$ is compatible with $x^2$?) than a prediction (what is $x^2$, given $x^1$?).

This is one of many example of interest, as summarized above. \MarginPar{above, that is, in introduction to batteries}
THE REST OF THIS COULD BE MOVED OR DELETED:

Having $p$ factored in this form allows implicit sampling to be applied to draw samples from this joint 
distribution.  
In this way statistics of pressure can incorporate models and measurements across scales.
We propose to further develop this example, to assess the validity of assumptions (1) and (2) above [IN THE 
FORMULATION SECTION], and
study the implementation of implicit sampling in this context.
This is likely a case where derivatives of models with respect to parameters is not directly possible
through adjoints.  But, the formulation involves parameters, not states, so the problems are not
high-dimensional.  We will pursue derivative free optimization (DFO) for equation (XXX) above in this case. 
\MarginPar{eqn XXX is meant to be the IS definition of F}

The use of DFO represents another research path applied to this example.  In particular,
many DFO algorithms utilize a ``population" of candidate solutions.  
(e.g., covariance matrix adaptation evolutionary strategy (CMA-ES) \cite{hansenrankmu,cmaes},
as well as of course traditional evolutionary algorithms such as genetic algorithms, etc).
There is thus an obvious connection to be explored between the set of particles being used in the overall
data assimilation problem, and the population (of particles, we may say) being used in the DFO.
Of special interest is the CMA-ES algorithm as it involves
 dynamically identifying scales on different axes and exploiting
this knowledge in optimization (i.e. converting long thin valleys of search space into bowls); this
is clearly related to ``affine sampling", discussed elsewhere in this
proposal. \MarginPar{where?}

\bibliography{batteries3}
\bibliographystyle{plain}


\end{document}
