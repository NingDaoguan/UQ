\documentclass[11pt]{article}
\setlength{\parindent}{0. true in}
%\pagestyle{empty}
\textwidth 6.25 truein
\oddsidemargin 0.1 truein
\evensidemargin .1 truein
\topmargin -.8 truein
\textheight 9.7 truein
\begin{document}


\begin{center}
{\large{\bf George Shu Heng Pau -- Biographical Sketch}} \\
\end{center}

\vspace{.2 true in}

George Shu Heng Pau is a research scientist within the Earth Science Division of Lawrence Berkeley National Laboratory (LBNL). He specializes in high-performance computation, reduced order modeling, inverse modeling, and uncertainty quantifications of subsurface flows. He is a key contributor to the Department of Energy's Advanced Simulation Capabilities for Environmental Management (ASCEM) project and the development of reduced-order models for the Department of Energy's National Risk Assessment Partnership project (NRAP).  Prior to joining the Earth Science Division, Pau's postdoctoral research focused on the development of parallel adaptive mesh refinement algorithms for multiphase subsurface flow, and his work is currently being used within the ASCEM project. For his doctoral dissertation at Massachusetts Institute of Technology, Pau developed reduced-order modeling techniques for nonlinear partial differential equations. 

\subsection*{Education and Training}
\begin{enumerate}
\setlength{\itemsep}{1pt}
\setlength{\parskip}{0pt}
\setlength{\parsep}{0pt}
\item B.Eng (Hons) in Mechanical Engineering, Petronas University of Technology, Malaysia. June 1998 -- May 2001. 
\item S.M degree in High Performance Computation in Engineered System, Singapore-MIT Alliance, National University of Singapore, Singapore. June 2001 -- July 2002
\item Ph.D. degree in Mechanical Engineering, Massachusetts Institute of Technology, Cambridge, MA. September 2003 -- June 2007. 
\item Luis W. Alvarez Postdoctoral Fellow in Computational Science, Computational Research Division, Lawrence Berkeley National Laboratory, Berkeley, CA. July 2007 -- January 2011. 
\end{enumerate}

\subsection*{Research and Professional Experience}
Career-track research scientist, Earth Science Division, Lawrence Berkeley National Laboratory, Berkeley, CA. January 2011 -- present. 

{\em Description}: In this position, Pau develops efficient optimization algorithms and model reduction methods for large-scale inverse modeling of subsurface flow.  This work is supported by the Department of Energy's National Risk Assessment Partnership project.  He also develops the simulator and the optimization software used in the Department of Energy's Advanced Simulation Capability for Environmental Management project.   

\subsection*{Publications}
\begin{enumerate}
\setlength{\itemsep}{1pt}
\setlength{\parskip}{0pt}
\setlength{\parsep}{0pt}
\item Pau, G. S. H., Zhang, Y. and Finsterle, S.: Reduced order models for many-query subsurface flow applications.  {\em Computational Geosciences}, 2013. DOI: 10.1007/s10596-013-9349-z.
\item Zhang, Y. and Pau, G. S. H.: Reduced order model development for CO$_2$ Storage in Brine Reservoirs. {\em NRAP-TRS-III-005-2012, NRAP Technical Report Series}, U.S. Department Energy, National Energy Technology Laboratory, Morgantown, WV, p 20, 2012.
\item Pau, G. S. H., Zhang, Y. and Finsterle, S.: Reduced order models for subsurface flow in iTOUGH2.  In {\em Proceedings of TOUGH Symposium 2012}, Lawrence Berkeley National Laboratory, Berkeley, California, September 17-19, 2012.
\item Pau, G. S. H., Almgren, A. S., Bell, J. B., Fagnan, K. M. and Lijewski, M. J.: An adaptive mesh refinement algorithm for compressible two-phase flow in porous media. {\em Computational Geosciences}, 16(3), 577-592, 2011.
\item Pau, G. S. H., Bell, J. B., Pruess, K., Almgren, A. S., Lijewski, M. J. and Zhang, K.: High resolution simulation and characterization of density-driven flow in CO$_2$ storage in saline aquifers. {\em Advances in Water Resources}, 33(4), 443�455, 2010.
\item Pau, G. S. H., Almgren, A. S., Bell, J. B. and Lijewski, M. J.: A parallel second-order adaptive mesh algorithm for incompressible flow in porous media. {\em Philosophical Transactions of the Royal Society A}, 367, 4633-4654, 2009.
\item Maday, Y., Nguyen, N. C., Patera, A. T. and Pau, G. S. H.: A general multipurpose interpolation procedure: the magic points. {\em Communications on Pure and Applied Analysis}, 8(1), 383-404, 2009.
\item Pau, G. S. H.: Reduced-basis method for nanodevices simulation. {\em Physical Review B}, 78, 155425, 2008.
\item Pau, G. S. H.: Reduced-basis method for band structure calculations. {\em Physical Review E}, 76, 046704, 2007.
\item Cances, E., Le Bris, C., Nguyen, N. C., Maday, Y., Patera, A. T. and Pau, G. S. H.: Feasibility and competitiveness of a reduced basis approach for rapid electronic structure calculations in quantum chemistry. In {\em High-Dimensional Partial Differential Equations in Science and Engineering}, CRM Proceedings Volume 41, AMS, 2007.
\end{enumerate}

\subsection*{Awards}
Early Career Research Award, U.S. Department of Energy, 2013. Director�s Award for Exceptional Achievement, Lawrence Berkeley National Laboratory, 2012. Luis W. Alvarez Postdoctoral Fellowship in Computational Science, Lawrence Berkeley National Laboratory, 2007.

\subsection*{Synergistic Activities}
Member of American Geophysical Union. Reviewer for the following journals: Water Resources Research, Physical Review Letters, Computational Geosciences, Computer and Geosciences, Proceedings of the Royal Society A, Special Topics and Reviews in Porous Media, International Journal of Heat Mass Transfer.

\subsection*{Collaborators and Co-editors}
Agarwal, D. (Lawrence Berkeley National Laboratory); Almgren, A.S. (Lawrence Berkeley National Laboratory); Cances, E. (Ecole des Ponts Paris Tech, France); Day, M. (Lawrence Berkeley National Laboratory); Fagnan, K.M. (Lawrence Berkeley National Laboratory); Finsterle, S. (Lawrence Berkeley National Laboratory); Gordon, I. (Carnegie Mellon University); Keating, E. (Los Alamos National Laboratory); Pruess, K. (Lawrence Berkeley National Laboratory); Le Bris, C. (Ecole des Ponts ParisTech, France); Lijeswki, M.J. (Lawrence Berkeley National Laboratory); Liu, G (University of Cincinnanti); Maday, Y (Universit� Paris VI - Pierre et Marie Curie, France.); Nguyen, N.C. (Massachusetts Institute of Technology); Schuchardt, K.L. (Pacific Northwest National Laboratory); Sonnenthal, E. (Lawrence Berkeley National Laboratory); Spycher, N. (Lawrence Berkeley National Laboratory); Vesselinov, V.V. (Los Alamos National Laboratory); Xu, T. (Lawrence Berkeley National Laboratory); Zhang, G. (Shell International Exploration \& Production Inc); Zhang, K (Lawrence Berkeley National Laboratory); Zhang, Y. (Lawrence Berkeley National Laboratory); Zheng, H. (Institute of High Performance Computing, Singapore).

\subsection*{Graduate and Postdoctoral Advisors}
\begin{enumerate}
\setlength{\itemsep}{1pt}
\setlength{\parskip}{0pt}
\setlength{\parsep}{0pt}
\item Bell, J.B. Lawrence Berkeley National Laboratory. Postdoctoral Advisor.
\item Patera, A.T.  Massachusetts Institute of Technology. Graduate Advisor.
\end{enumerate}
\end{document}
