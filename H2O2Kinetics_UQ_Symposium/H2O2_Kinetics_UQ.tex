 \documentclass[preprint,3p,times,twocolumn]{elsarticle}
% \documentclass[review,1p,times]{elsarticle} % For submission

%\documentclass{CRPITStyle} 
%%\usepackage{epsfig}   % Packages to use if you wish
%%\usepackage{lscape}   % 
%\usepackage[authoryear]{natbib}
%\renewcommand{\cite}{\citep}
%\pagestyle{empty}
%\thispagestyle{empty}
%\hyphenation{roddick}

\usepackage{booktabs}

\usepackage{graphicx}
\usepackage{amssymb}
%% \usepackage{lineno}
\usepackage{hanging}
%\renewcommand{\mag}[1]{\ensuremath{|{#1}|}}
\newcommand{\mf}{\xi}
\newcommand{\sd}{\chi}
\journal{Proceedings of The Combustion Institute}




\begin{document}

\begin{frontmatter}

%% Title, authors and addresses

%% use the tnoteref command within \title for footnotes;
%% use the tnotetext command for the associated footnote;
%% use the fnref command within \author or \address for footnotes;
%% use the fntext command for the associated footnote;
%% use the corref command within \author for corresponding author footnotes;
%% use the cortext command for the associated footnote;
%% use the ead command for the email address,
%% and the form \ead[url] for the home page:
%%
%% \title{Title\tnoteref{label1}}
%% \tnotetext[label1]{}
%% \author{Name\corref{cor1}\fnref{label2}}
%% \ead{email address}
%% \ead[url]{home page}
%% \fntext[label2]{}
%% \cortext[cor1]{}
%% \address{Address\fnref{label3}}
%% \fntext[label3]{}


\title{Determining the joint distribution of hydrogen kinetic
  mechanism parameters consistent with macroscopic experiments}
\author{}



\begin{abstract}
  
\end{abstract}

\begin{keyword}
  MCMC, hydrogen kinetics, parameter estimation, uncertainty
  quantification
\end{keyword}

\end{frontmatter}

\begin{abstract}
\end{abstract}
\vspace{.1in}

\noindent {\em Keywords:} 

\section{Introduction}
Hydrogen-oxygen kinetics have been the subject of extensive study,
both because of interest in hydrogen and hydrogen-enriched fuels and
because the $H_2$-$O_2$ mechanism forms an important submechanism in
hydrocarbon kinetic mechanisms. Two primary datasources have led to
the current state-of-the-art hydrogen oxidation mechanisms. Firstly,
elementary rate measurements, carefully constructed experiments
sensitive to one or a small number of individual reaction rates are
used to infer individual rate parameters such as done by
\cite{MuellerYD98, AshmanH98}. Secondly, macro experiments where the
observable depends on the entire reaction mechanism have been used to
refine mechanisms rates in a `comprehensive' sense. Such mechanisms
have been developed by Mueller et al.  \cite{YetterDR91, MuellerKYD99}
and subsqeuently refined by Li et al.  \cite{LiZKD04}, O'Conaire et
al., \cite{OConaireCSPW04}. The Li et al.  mechanism has since been
updated by Burke et al. \cite{BurkeCDJ10, BurkeCJDK11} while the
O'Conaire et al. mechanism has since been updated by Keromnes et
al. \cite{Keromnes_Many_Curran13}. A similar kinetic mechanism has
been developed by Konnov \cite{Konnov08}, also by assembling
elementary reactions from the literature.

Closely related developments have been made for kinetic models
involving hydrogen and carbon dioxide (syngas). Li et
al. \cite{LiZKCDS07} augmented their earlier $H_2/O_2$ mechanisms with
$C_1/O_2$ kinetics and then adjusted key rate constants identified by
sensitivity analysis to improve predictions of the macro
experiments. Davis et al.  \cite{DavisJWE05} placed somewhat heavier
emphasis on macro optimization and developed a $H_2/CO$ mechanism
based on assembling a trial mechanism drawing significantly the
$H_2/C_1/O_2$ chemistry from GRIMech \cite{GRIMech} and then
optimizing a significant number of kinetics parameters against a
library of laminar flame, flow reactor and shock-tube experiments. You
et al. \cite{YouPF11} produced a viable hydrogen mechanism as a
byproduct of establishing the `data collaboration' method as part of a
workflow to automate mechanism generation. The mechanism was
subsequently the focus of an uncertainty analysis study performed by
Li et al. \cite{LiYWL15}.

While such global optimization of kinetic mechanisms is typically done
within the limits of the uncertainty bounds on coefficients known from
elementary rate measurements, the technique has come under criticism
for resulting in mechanisms that are inconsistent with the more
fundamental measurements \cite{DavisJWE05_comment}. We also find it
troubling that, despite extensive experimental databases, a typical
end-use of such comprehensive oxidation mechanisms is in situations
such as turbulent flame calculations that may exercise parts of the
state space not well covered by 0D and 1D experiments.

Braman et al. \cite{BramanOR13} used Bayesian methods to quantify the
distribution of parameters in several syngas mechanisms consistent
with a database of premixed laminar flames. The challenge with
Bayesian methods is that typical sampling methods, such as used by
Braman et al., are expensive. Using informative priors to use
presupposed knowledge about the posterior distribution reduces the
state space that must be explored at the cost of results that depend
on the prior distribution used.

Here, we focus on Bayesian analysis of the kinetics parameters in a
hydrogen oxidation mechanism that are not well constrained by
elementary measurements. We employ a sophisticated sampling
methodology that, combined with high performance computing, makes it
tractable to use broad, `uniformative' priors to determine the
distribution of parameters consistent with a large database of macro
experimental measurements. Our contributions are several. Firstly, we
provide a joint distribution of the kinetics parameters that are
consistent with the macro experiments.  This distribution is useful
for forward-sampling in combustion calculations to facilitate
uncertainty quantification. Secondly, we identify areas of degeneracy
where the suite of macro experiments is insufficient to distinguish
between alternative combinations of parameters, suggesting
opportunities where the effort to conduct elementary rate measurements
would be most valuable. Thirdly, and somewhat incidentally, we have
produced a suite of candidate mechanisms well suited for lean, high
pressure flames, some of which exhibit significantly improved
agreement with the available experimental data than the starting point
for our sampling procedure. Finally, this effort demonstrates a
procedure for using Bayesian analysis as a framework for combining
simulation and experimental data in a systematic way to identify the
underlying physical characteristics of the system being investigated.

\section{Methodology}
In a Bayesian framework, the posterior distribution is \ldots

\begin{itemize}
    \item emcee sampler. Note key features: affine invariant (good for sampling a banana), 
    multiple walkers facilitate parallel over samples. Reduces autocorrelation time. 
    \item Start from MLE? We didn't... we started from best guess from literature. 
    \item How did we choose ensemble std?
    \item Assign $\sigma$ for each experiment based on experimental uncertainty
    \item Parallelize over experiments and red-black over walkers
    \item use chemkin III + boxlib to build up evaluator that can vary
      parameters efficiently and run in parallel.
    \item set up reasonably robust solution procedure; restart from
      common baseline solution, refine, ramp pressure. Failure to
      converge returned as zero likelyhood.
    \item uninformative, broad priors. uniform within bounds; bounds
      wided and restarted until sampler not running into bounds
    \item Don't specifically include separate elementary rate studies
        \begin{itemize}
        \item Encapsulate `what parameter set is consistent with macro
          data'
        \item Leave parameters well understood (or at well studied)
          fixed
        \item Elementary rate data could be use dot eliminate
          ambiguity in strongly correlated dimensions; if not
          existing, suggests what coudl be done. Method allows
          discovery of parameter combinations that the set of macro
          experiments does not distinguish between.
 \end{itemize}
\end{itemize}
\subsection{Mechanism}
\begin{table}[htp]
  \centering
  \begin{tabular}{l l l l}
\hline
(R0)&$ \mathrm{H} + \mathrm{O}_2 = \mathrm{O} + \mathrm{OH}$\\
(R1)&$ \mathrm{H}_2 + \mathrm{O} = \mathrm{H} + \mathrm{OH}$\\
(R2)&$ \mathrm{H}_2 + \mathrm{O} = \mathrm{H} + \mathrm{OH}$\\
(R3)&$ \mathrm{H}_2 + \mathrm{OH} = \mathrm{H} + \mathrm{H}_2\mathrm{O}$\\
(R4)&$ 2\mathrm{OH} = \mathrm{O} + \mathrm{H}_2\mathrm{O}$\\
(R5)&$ \mathrm{H}_2 = 2\mathrm{H}$\\
(R6)&$ \mathrm{H}_2 = 2\mathrm{H}$\\
(R7)&$ \mathrm{H}_2 = 2\mathrm{H}$\\
(R8)&$ 2\mathrm{O} = \mathrm{O}_2$\\
(R9)&$ 2\mathrm{O} = \mathrm{O}_2$\\
(R10)&$ 2\mathrm{O} = \mathrm{O}_2$\\
(R11)&$ \mathrm{H} + \mathrm{O} = \mathrm{OH}$\\
(R12)&$ \mathrm{H}_2\mathrm{O} = \mathrm{H} + \mathrm{OH}$\\
(R13)&$ \mathrm{H}_2\mathrm{O} = \mathrm{H} + \mathrm{OH}$\\
(R14)&$ \mathrm{H} + \mathrm{O}_2 = \mathrm{HO}_2$\\
(R15)&$ \mathrm{H} + \mathrm{HO}_2 = \mathrm{H}_2 + \mathrm{O}_2$\\
(R16)&$ \mathrm{H} + \mathrm{HO}_2 = 2\mathrm{OH}$\\
(R17)&$ \mathrm{O} + \mathrm{HO}_2 = \mathrm{OH} + \mathrm{O}_2$\\
(R18)&$ \mathrm{OH} + \mathrm{HO}_2 = \mathrm{H}_2\mathrm{O} + \mathrm{O}_2$\\
(R19)&$ 2\mathrm{HO}_2 = \mathrm{O}_2 + \mathrm{H}_2\mathrm{O}_2$\\
(R20)&$ 2\mathrm{HO}_2 = \mathrm{O}_2 + \mathrm{H}_2\mathrm{O}_2$\\
(R21)&$ \mathrm{H}_2\mathrm{O}_2 = 2\mathrm{OH}$\\
(R22)&$ \mathrm{H} + \mathrm{H}_2\mathrm{O}_2 = \mathrm{OH} + \mathrm{H}_2\mathrm{O}$\\
(R23)&$ \mathrm{H} + \mathrm{H}_2\mathrm{O}_2 = \mathrm{H}_2 + \mathrm{HO}_2$\\
(R24)&$ \mathrm{O} + \mathrm{H}_2\mathrm{O}_2 = \mathrm{OH} + \mathrm{HO}_2$\\
(R25)&$ \mathrm{OH} + \mathrm{H}_2\mathrm{O}_2 = \mathrm{H}_2\mathrm{O} + \mathrm{HO}_2$\\
(R26)&$ \mathrm{OH} + \mathrm{H}_2\mathrm{O}_2 = \mathrm{H}_2\mathrm{O} + \mathrm{HO}_2$\\
(R27)&$ \mathrm{H} + \mathrm{HO}_2 = \mathrm{O} + \mathrm{H}_2\mathrm{O}$\\
(R28)&$ \mathrm{O} + \mathrm{OH} = \mathrm{HO}_2$\\
\hline
  \end{tabular}
  \caption{Starting Mechanism}
  \label{tab:srcmech}
\end{table}
\begin{itemize}
\item Start with Burke et al. mechanism. Comes from Li et al.,
  assembly of elementary rates, minor targeted optimization.
\item Augmented with some extra reactions identified by Burke (X1,
  X6). Put in a table.
\item Vary those parameters deemed relevant for high pressure, lean,
  not well understood: HO2, H2O2 + associated third body efficiencies
\item Need to note and be upfront that we are varying the parameters
  for rxns that form half of a delicate balance; other half determined
  by elementary rate expressions
\item Need a (compact) table showing the mechanism and parameters
  varied. Maybe model after the one in the Davis paper.
  \begin{table}[htp]
    \centering
    \begin{tabular}{l l l}
      \hline
(P0)&$ (R14): \mathrm{H} + \mathrm{O}_2 = \mathrm{HO}_2     $&$ \mathrm{A}$\\
(P1)&$ (R14): \mathrm{H} + \mathrm{O}_2 = \mathrm{HO}_2     $&$ \mathrm{A}_\mathrm{low}$\\
(P2)&$ (R14): \mathrm{H} + \mathrm{O}_2 = \mathrm{HO}_2     $&$ \mathrm{M} \mathrm{H}_2$\\
(P3)&$ (R14): \mathrm{H} + \mathrm{O}_2 = \mathrm{HO}_2     $&$ \mathrm{M} \mathrm{H}_2\mathrm{O}$\\
(P4)&$ (R14): \mathrm{H} + \mathrm{O}_2 = \mathrm{HO}_2     $&$ \mathrm{M} \mathrm{O}_2$\\
(P5)&$ (R14): \mathrm{H} + \mathrm{O}_2 = \mathrm{HO}_2     $&$ \mathrm{M} \mathrm{A}R$\\
(P6)&$ (R14): \mathrm{H} + \mathrm{O}_2 = \mathrm{HO}_2     $&$ \mathrm{M} \mathrm{HE}$\\
(P7)&$ (R15): \mathrm{H} + \mathrm{HO}_2 = \mathrm{H}_2 + \mathrm{O}_2     $&$ \mathrm{A}$\\
(P8)&$ (R16): \mathrm{H} + \mathrm{HO}_2 = 2\mathrm{OH}     $&$ \mathrm{A}$\\
(P9)&$ (R17): \mathrm{O} + \mathrm{HO}_2 = \mathrm{OH} + \mathrm{O}_2     $&$ \mathrm{A}$\\
(P10)&$ (R18): \mathrm{OH} + \mathrm{HO}_2 = \mathrm{H}_2\mathrm{O} + \mathrm{O}_2     $&$ \mathrm{A}$\\
(P11)&$ (R19): 2\mathrm{HO}_2 = \mathrm{O}_2 + \mathrm{H}_2\mathrm{O}_2     $&$ \mathrm{A}$\\
(P12)&$ (R20): 2\mathrm{HO}_2 = \mathrm{O}_2 + \mathrm{H}_2\mathrm{O}_2     $&$ \mathrm{A}$\\
(P13)&$ (R21): \mathrm{H}_2\mathrm{O}_2 = 2\mathrm{OH}     $&$ \mathrm{A}$\\
(P14)&$ (R21): \mathrm{H}_2\mathrm{O}_2 = 2\mathrm{OH}     $&$ \mathrm{A}_\mathrm{low}$\\
(P15)&$ (R21): \mathrm{H}_2\mathrm{O}_2 = 2\mathrm{OH}     $&$ \mathrm{M} \mathrm{H}_2\mathrm{O}$\\
(P16)&$ (R21): \mathrm{H}_2\mathrm{O}_2 = 2\mathrm{OH}     $&$ \mathrm{M} \mathrm{H}_2\mathrm{O}_2$\\
(P17)&$ (R21): \mathrm{H}_2\mathrm{O}_2 = 2\mathrm{OH}     $&$ \mathrm{M} \mathrm{O}_2$\\
(P18)&$ (R21): \mathrm{H}_2\mathrm{O}_2 = 2\mathrm{OH}     $&$ \mathrm{M} \mathrm{N}_2$\\
(P19)&$ (R21): \mathrm{H}_2\mathrm{O}_2 = 2\mathrm{OH}     $&$ \mathrm{M} \mathrm{H}_2$\\
(P20)&$ (R21): \mathrm{H}_2\mathrm{O}_2 = 2\mathrm{OH}     $&$ \mathrm{M} \mathrm{HE}$\\
(P21)&$ (R22): \mathrm{H} + \mathrm{H}_2\mathrm{O}_2 = \mathrm{OH} + \mathrm{H}_2\mathrm{O}     $&$ \mathrm{A}$\\
(P22)&$ (R23): \mathrm{H} + \mathrm{H}_2\mathrm{O}_2 = \mathrm{H}_2 + \mathrm{HO}_2     $&$ \mathrm{A}$\\
(P23)&$ (R24): \mathrm{O} + \mathrm{H}_2\mathrm{O}_2 = \mathrm{OH} + \mathrm{HO}_2     $&$ \mathrm{A}$\\
(P24)&$ (R25): \mathrm{OH} + \mathrm{H}_2\mathrm{O}_2 = \mathrm{H}_2\mathrm{O} + \mathrm{HO}_2     $&$ \mathrm{A}$\\
(P25)&$ (R26): \mathrm{OH} + \mathrm{H}_2\mathrm{O}_2 = \mathrm{H}_2\mathrm{O} + \mathrm{HO}_2     $&$ \mathrm{A}$\\
(P26)&$ (R27): \mathrm{H} + \mathrm{HO}_2 = \mathrm{O} + \mathrm{H}_2\mathrm{O}     $&$ \mathrm{A}$\\
(P27)&$ (R28): \mathrm{O} + \mathrm{OH} = \mathrm{HO}_2     $&$ \mathrm{A}$\\
(P28)&$ (R28): \mathrm{O} + \mathrm{OH} = \mathrm{HO}_2     $&$ \mathrm{M} \mathrm{H}_2$\\
(P29)&$ (R28): \mathrm{O} + \mathrm{OH} = \mathrm{HO}_2     $&$ \mathrm{M} \mathrm{H}_2\mathrm{O}$\\
(P30)&$ (R28): \mathrm{O} + \mathrm{OH} = \mathrm{HO}_2     $&$ \mathrm{M} \mathrm{A}R$\\
\hline
\end{tabular}
    \caption{Active parameters}
    \label{tab:parameters}
  \end{table}
\end{itemize}
\subsection{Experimental Targets}
\begin{itemize}
\item Combination of those from Davis (flw, flw, ign) and Burke
  (flames)
\item Probably need a table with original source.
\end{itemize}
\section{Results}
Figures:
\begin{enumerate}
\item Triangle plot (1pg)
\item Flamespeed vs. equivalence ratio, p= 1, 5, 10 atm for starting
  mechanism, literature mechanisms (Li, Saxena, Konnov, O'Conaire,
  Burke), and distribution by sampling from our posterior
\item Same but vs P, $\phi=0.3,0.5$; $T_f=1400$
    \item Interesting conditional, marginal pdfs
    \item Choose several points with equivalent goodness of fit, show
      experimental agreement. Possibly show detailed radical profiles
      for 1d flame, esp. if they differ significantly. Note that if
      such profiles could be measured, could discriminate between
      otherwise confounded mechanisms. Alternatively, show plot
      experimental data points and equivalent mechanism a, b, c,
      \ldots
 \end{enumerate}
 
 Notes --- see if supported by data:
 \begin{itemize}
    \item Parameter 2 dominates all others, except maybe 8,9, 10, 22
    \item Correlation between P8, P9
    \item Positive correlation between P8, P12
    \item Negative correlation between P4, P7  
    \item etc... check for correlations on converged sampling
    \item Marginal distributions - Normal or not? Cf. to Braman et al.
 \end{itemize}
%{\scriptsize
%\begin{tabular}{lllrrrrrrrrr}
\toprule
{} & Bath &         Ref &     T0 &  Target &     Tf &  XH2 &  XHE &  XO2 &   p0 &  phi &  sigma \\
\midrule
ID            &      &             &        &         &        &      &      &      &      &      &        \\
BurkeCDJ10\_0  &  NaN &  BurkeCDJ10 &  295.0 &    95.3 & 1600.0 &  0.1 &  0.8 &  0.1 &  1.0 &  0.9 &   10.4 \\
BurkeCDJ10\_1  &  NaN &  BurkeCDJ10 &  295.0 &    77.4 & 1600.0 &  0.1 &  0.8 &  0.1 &  5.0 &  0.9 &   12.8 \\
BurkeCDJ10\_2  &  NaN &  BurkeCDJ10 &  295.0 &    45.9 & 1600.0 &  0.1 &  0.8 &  0.1 & 10.0 &  0.9 &    4.9 \\
BurkeCDJ10\_3  &  NaN &  BurkeCDJ10 &  295.0 &    28.3 & 1600.0 &  0.1 &  0.8 &  0.1 & 15.0 &  0.9 &    2.5 \\
BurkeCDJ10\_4  &  NaN &  BurkeCDJ10 &  295.0 &    20.0 & 1600.0 &  0.1 &  0.8 &  0.1 & 20.0 &  0.9 &    1.6 \\
BurkeCDJ10\_5  &  NaN &  BurkeCDJ10 &  295.0 &    15.2 & 1600.0 &  0.1 &  0.8 &  0.1 & 25.0 &  0.9 &    1.2 \\
BurkeCDJ10\_6  &  NaN &  BurkeCDJ10 &  295.0 &    83.0 & 1500.0 &  0.1 &  0.8 &  0.1 &  1.0 &  1.0 &   13.8 \\
BurkeCDJ10\_7  &  NaN &  BurkeCDJ10 &  295.0 &    48.7 & 1500.0 &  0.1 &  0.8 &  0.1 &  5.0 &  1.0 &    9.2 \\
BurkeCDJ10\_8  &  NaN &  BurkeCDJ10 &  295.0 &    27.3 & 1500.0 &  0.1 &  0.8 &  0.1 & 10.0 &  1.0 &    3.6 \\
BurkeCDJ10\_9  &  NaN &  BurkeCDJ10 &  295.0 &    87.0 & 1600.0 &  0.1 &  0.8 &  0.1 &  1.0 &  1.0 &    9.2 \\
BurkeCDJ10\_10 &  NaN &  BurkeCDJ10 &  295.0 &    61.9 & 1600.0 &  0.1 &  0.8 &  0.1 &  5.0 &  1.0 &   10.8 \\
BurkeCDJ10\_11 &  NaN &  BurkeCDJ10 &  295.0 &    40.7 & 1600.0 &  0.1 &  0.8 &  0.1 & 10.0 &  1.0 &    4.5 \\
BurkeCDJ10\_12 &  NaN &  BurkeCDJ10 &  295.0 &    28.7 & 1600.0 &  0.1 &  0.8 &  0.1 & 15.0 &  1.0 &    2.7 \\
BurkeCDJ10\_13 &  NaN &  BurkeCDJ10 &  295.0 &    21.4 & 1600.0 &  0.1 &  0.8 &  0.1 & 20.0 &  1.0 &    1.8 \\
BurkeCDJ10\_14 &  NaN &  BurkeCDJ10 &  295.0 &    14.5 & 1600.0 &  0.1 &  0.8 &  0.1 & 25.0 &  1.0 &    1.1 \\
BurkeDJ11\_5   &  NaN &   BurkeDJ11 &  295.0 &    61.0 & 1400.0 &  0.1 &  0.8 &  0.1 &  1.0 &  0.5 &    3.6 \\
BurkeDJ11\_6   &  NaN &   BurkeDJ11 &  295.0 &    37.9 & 1400.0 &  0.1 &  0.8 &  0.1 &  2.5 &  0.5 &    4.4 \\
BurkeDJ11\_7   &  NaN &   BurkeDJ11 &  295.0 &    23.1 & 1400.0 &  0.1 &  0.8 &  0.1 &  5.0 &  0.5 &    2.2 \\
BurkeDJ11\_8   &  NaN &   BurkeDJ11 &  295.0 &    15.0 & 1400.0 &  0.1 &  0.8 &  0.1 &  7.5 &  0.5 &    1.5 \\
BurkeDJ11\_9   &  NaN &   BurkeDJ11 &  295.0 &     8.2 & 1400.0 &  0.1 &  0.8 &  0.1 & 10.0 &  0.5 &    0.7 \\
BurkeDJ11\_10  &  NaN &   BurkeDJ11 &  295.0 &   105.1 & 1600.0 &  0.1 &  0.8 &  0.1 &  1.0 &  0.5 &    3.4 \\
BurkeDJ11\_11  &  NaN &   BurkeDJ11 &  295.0 &    64.3 & 1600.0 &  0.1 &  0.8 &  0.1 &  5.0 &  0.5 &    2.7 \\
BurkeDJ11\_12  &  NaN &   BurkeDJ11 &  295.0 &    40.1 & 1600.0 &  0.1 &  0.8 &  0.1 & 10.0 &  0.5 &    1.7 \\
BurkeDJ11\_13  &  NaN &   BurkeDJ11 &  295.0 &    59.7 & 1400.0 &  0.1 &  0.8 &  0.1 &  1.0 &  0.7 &    4.3 \\
BurkeDJ11\_14  &  NaN &   BurkeDJ11 &  295.0 &    42.3 & 1400.0 &  0.1 &  0.8 &  0.1 &  2.5 &  0.7 &    3.4 \\
BurkeDJ11\_15  &  NaN &   BurkeDJ11 &  295.0 &    27.6 & 1400.0 &  0.1 &  0.8 &  0.1 &  5.0 &  0.7 &    1.7 \\
BurkeDJ11\_16  &  NaN &   BurkeDJ11 &  295.0 &    17.7 & 1400.0 &  0.1 &  0.8 &  0.1 &  7.5 &  0.7 &    1.1 \\
BurkeDJ11\_17  &  NaN &   BurkeDJ11 &  295.0 &    10.3 & 1400.0 &  0.1 &  0.8 &  0.1 & 10.0 &  0.7 &    0.8 \\
BurkeDJ11\_18  &  NaN &   BurkeDJ11 &  295.0 &    97.7 & 1600.0 &  0.1 &  0.8 &  0.1 &  1.0 &  0.7 &    3.9 \\
BurkeDJ11\_19  &  NaN &   BurkeDJ11 &  295.0 &    67.0 & 1600.0 &  0.1 &  0.8 &  0.1 &  5.0 &  0.7 &    3.2 \\
BurkeDJ11\_20  &  NaN &   BurkeDJ11 &  295.0 &    44.3 & 1600.0 &  0.1 &  0.8 &  0.1 & 10.0 &  0.7 &    2.0 \\
BurkeDJ11\_21  &  NaN &   BurkeDJ11 &  295.0 &    31.4 & 1600.0 &  0.1 &  0.8 &  0.1 & 15.0 &  0.7 &    2.1 \\
BurkeDJ11\_22  &  NaN &   BurkeDJ11 &  295.0 &    22.4 & 1600.0 &  0.1 &  0.8 &  0.1 & 20.0 &  0.7 &    1.4 \\
BurkeDJ11\_23  &  NaN &   BurkeDJ11 &  295.0 &    16.3 & 1600.0 &  0.1 &  0.8 &  0.1 & 25.0 &  0.7 &    1.9 \\
BurkeDJ11\_33  &  NaN &   BurkeDJ11 &  295.0 &    95.3 & 1600.0 &  0.1 &  0.8 &  0.1 &  1.0 &  0.9 &   12.4 \\
BurkeDJ11\_34  &  NaN &   BurkeDJ11 &  295.0 &    66.2 & 1600.0 &  0.1 &  0.8 &  0.1 &  5.0 &  0.9 &    3.4 \\
BurkeDJ11\_35  &  NaN &   BurkeDJ11 &  295.0 &    87.0 & 1600.0 &  0.1 &  0.8 &  0.1 &  1.0 &  1.0 &    9.2 \\
BurkeDJ11\_36  &  NaN &   BurkeDJ11 &  295.0 &    61.9 & 1600.0 &  0.1 &  0.8 &  0.1 &  5.0 &  1.0 &   10.8 \\
BurkeDJ11\_37  &  NaN &   BurkeDJ11 &  295.0 &    40.7 & 1600.0 &  0.1 &  0.8 &  0.1 & 10.0 &  1.0 &    4.5 \\
fls1a         &   N2 &     [27–31] &  298.2 &   204.0 &    nan & 29.6 &  0.0 & 14.8 &  1.0 &  1.0 &   20.0 \\
fls2a         &   He &        [27] &  298.2 &   206.0 &    nan & 20.0 &  0.0 & 10.0 &  1.0 &  1.0 &   20.0 \\
fls3a         &   He &        [27] &  298.2 &    58.0 &    nan & 13.8 &  0.0 &  6.9 & 15.0 &  1.0 &    6.0 \\
flw1a         &   N2 &        [13] &  914.0 &     8.6 &    nan &  1.2 &  0.0 &  0.6 & 15.7 &  1.0 &    0.5 \\
flw2a         &   N2 &        [13] &  935.0 &    13.0 &    nan &  1.0 &  0.0 &  0.5 &  6.0 &  1.0 &    0.5 \\
flw3a         &   N2 &        [13] &  943.0 &    98.0 &    nan &  1.0 &  0.0 &  1.5 &  2.5 &  0.3 &   10.0 \\
flw4a         &   N2 &        [13] &  880.0 &   261.0 &    nan &  0.5 &  0.0 &  0.5 &  0.3 &  0.5 &   20.0 \\
flw8a         &   N2 &        [13] &  934.0 &    19.0 &    nan &  0.9 &  0.0 &  0.5 &  3.0 &  1.0 &    1.0 \\
flw8b         &   N2 &        [13] &  934.0 &    70.0 &    nan &  0.9 &  0.0 &  0.5 &  3.0 &  1.0 &    5.0 \\
ign2a         &   Ar &        [37] & 1033.0 &   238.0 &    nan & 20.0 &  0.0 & 10.0 &  0.5 &  1.0 &  140.0 \\
ign2b         &   Ar &        [37] & 1510.0 &    29.0 &    nan & 20.0 &  0.0 & 10.0 &  0.5 &  1.0 &   20.0 \\
ign3a         &   Ar &        [38] & 1754.0 &    10.0 &    nan &  0.5 &  0.0 &  0.2 & 33.0 &  1.0 &    3.0 \\
ign4a         &   Ar &        [38] & 1189.0 &   293.0 &    nan &  2.0 &  0.0 &  1.0 & 33.0 &  1.0 &  100.0 \\
ign4b         &   Ar &        [38] & 1300.0 &    11.0 &    nan &  2.0 &  0.0 &  1.0 & 33.0 &  1.0 &    4.0 \\
ign5a         &   Ar &        [38] & 1524.0 &    54.0 &    nan &  0.1 &  0.0 &  0.1 & 64.0 &  1.0 &   25.0 \\
\bottomrule
\end{tabular}

%}

\section{Conclusion}


\section{References}
\bibliographystyle{elsarticle-num-names}    
\bibliography{refs/hydrogen_oxidation,refs/chemistry_uq}

Evolution of hydrogen oxidation mechanisms:
        \begin{description}
            \item[Early work]
              \begin{itemize}
              \item \citet{WangR93} reduced mechanism (to 1 step) for
                CO oxidation, starting from full mechanism published
                in \citet{PetersR93}.
              \item$^S$ \citet{KimYD94}  CO mechanism and experiments; mechanism is
                revision to \citet{YetterDR91}. Flow reactor. Data for $\phi=0.3,1$
                up to 10atm, $\sim 1100k$. (RG: There is a comment by
                \cite{Konnov08} that there was a misprint in Yetter et
                al. paper, we should follow up if we use any Yetter et
                al. data to see if the misprint makes a difference to us)
              \item  \citet{TsangH86} source of many coefficients used
                in mechanism assembled by \citet{MuellerKYD99}. Experimental
                data from \citet{AungHF97, AungHF98}.
                \item \citet{KwonF01} Flame/stretch
                  interaction. Pressure exploration $<3$ atm.
                  \item \citet{TseZL00} published experimental data on
                    laminar flame speeds for $H_2/O_2/diluent$ based
                    on spherical expanding flame. Complementary
                    simulations based on Mueller mechanism. $0.5 <
                    \phi < 5$, $1 < p < 20atm$.
              \end{itemize}
                
            \item[2004] \citet{LiZKD04} publishes a revision of the Mueller et al. mechanism, by modifying:
                \begin{enumerate}
                    \item OH enthalpy of formation
                    \item Rate constant of the chain-branching reaction $H+O_2 = OH + O$, to match that suggested by Hessler.
                    \item The Troe formulation was applied to add the low
                        pressure limit for chain-termination reaction $H + O_2 (+M)
                        = HO_2 (+M)$ 
                    \item The rate constant of $H + OH + M = H_2O + M$
                        to match laminar flame speed predictions. Li notes that the
                        reported values of this constant span an order of magnitude
                        and that ``laminar flame speed predictions using any
                        particular set of diffusion coefficients recommended by
                        various authors can be forced dot predict the same flame
                        speed simply by adjusting the value of this single rate
                        constant''.
                \end{enumerate}
                \item[2004] \citet{ConaireCSPW04}. Published revised
                  mechanism with the same 19 reactions as the
                  near-simultaneously published Li mechanism. Differences:
                  \begin{enumerate}
                  \item Coefficients for reactions 1,8, 17
                    \item No bath gas dependent duplicate reactions
                      for reaction 5, 6
                      \item No Troe formulation for reactions 9, 15
                  \end{enumerate}
            \item[2005]$^S$ \citet{DavisJWE05} Optimization of Mueller
                mechanism against at library of shock tube, laminar flame
                experiments. Mostly moderate pressure (1-5atm), stoichiometric
                to rich
            \item[2006] \citet{SaxenaW06} extracted the
              hydrogen/syngas sub-mechanism from their propane
              mechanism  \citet{PetrovaW06} (check citation)
                \begin{enumerate}
                    \item Tested their sub-mechanism against data from
                      \citet{TseZL00} and \citet{KwonF01}
                      \item Also tested against diffusion flame
                        extinction data from \citet{BalakrishnanTW94}
                    \item Reference \citet{BalakrishnanW94} as ancestor of mechanism
                    \item Reference \citet{LiZKD04}, \citet{ConaireCSPW04}, and
                      \citet{DavisJWE05} as `newer' literature supporting
                        tweaking some of the rate parameters
                \end{enumerate}
                \item[2006] \citet{FrassoldatiFR06}, primarily
                  interested in nitrogen chemistry and $NO_x$
                  formation in hydrogen combustion. 
                  \begin{enumerate}
                    \item Starting mechanism \citet{RanziFGF05}.
                    \item Experimental data for $H_2$/air premixed
                      flames \cite{HarringtonSBNJC96}, stirred
                      reactor \cite{XieHH96}. 
                  \item Adjusted rate parameters for $H + OH + M =
                    H_2O$ to improve predictions of high pressure
                    flames. 
                    \item In appendix, note results sensitive to
                      competition between $H + O_2 + M = HO_2 + M$ and
                      $H+O2 = OH + O$, especially starting at about 3
                      atm
                      \item In low pressure sweep of temperature found
                        Li mechanism, Mass and Warnatz mechanism and
                        their tweaked mechanism worked pretty
                        well. Konnov, O Conaire, previous Ranzi
                        mechanism not so well. 
                        \item Note bath gas effects due to flame
                          temperatures, Chaperon efficiencies and
                          transport properties.
                        \item Ref experiments adding water to
                          autoignition of hydrogen flaems in
                          counterflow. Note autoignition temperature
                          in this case extremely sensitive to third
                          body efficiency of $H_2O$ in the $H + O_2 +
                          M = HO_2 +M$ reaction.
                  \end{enumerate}
                  \item[2007] \citet{LiZKCDS07} Li mechanism for $C_1$ chemistry
            \item[2007] \citet{StrohleM07} examine $H_2$ mechanisms
              under gas turbine conditions
              \item[2008] \citet{Konnov08}. References to updated
                mechanism of \citet{Konnov04}, from \citet{Konnov00}. Mechanism has 21
                reactions.  Need to read carefully, appears to be
                mechanism with separate geneology. 
            \item[2010]$^S$ \citet{BurkeCDJ10} collection of experimental data
                for high pressure (1-25atm), low temperature flames for $\phi
                \ge 0.85$. Expanding spherical flames in pressure vessel with
                optical access. 
                \begin{enumerate}
                    \item Group most sensitive reactions as: their R5-R7, R1,
                        which are sensitive under all conditions, including
                        those studied; $HO_2 + R$ reactions, as major pathways
                        at high pressures when $HO_2$ concentrations are
                        higher, and radical-radical recombination reactions
                        such as $H + OH + M = H_2O + M$,  $O + OH + M = HO_2 +
                        M$ and $H + HO_2 + M = H_2O_2 + M$, important in high
                        pressure flames where concentration of radicals and
                        collision partners is higher. 
                    \item Note that existing mechanisms fail to predict high pressure fall off
                    \item Note $O + OH + M = HO_2 + M$ demonstrated to
                        significantly impact lean high pressure flames and
                        frequently absent. 
                    \item High pressure flames much more sensitive to recombination reactions
                    \item Pressure shifts flames structure, makes branching activity narrower
                \end{enumerate}
            \item[2011] \citet{YouPF11} establish data collaboration method
                \begin{enumerate}
                    \item Explicitly, focus not on building a better hydrogen mechanism, but on establishing a workflow to automate generation of mechanisms
                    \item Assembled trial model of 10 species and 21 reactions 
                    \item Rate coefficients taken form GRImech, Tsang and Thompson, Davis, Hong, Baulch, Michael, Saxena and Williams, Burke`
                \end{enumerate}
                        
            \item[2011 (2010)] \citet{BurkeCJDK12} publish and update to the
                2004 Li mechanism with an emphasis on high-pressure,
                low-temperature flames
                \begin{enumerate}
                    \item $H + HO_2 = H_2O +O$ might be necessary
                    \item Bath gas mixing rule improvement for $H + O_2 (+M) =
                        HO_2 (+M)$ might be necessary for multicomponent bath
                        gases at high pressure
                    \item Kinetic regime relevant to gas turbine combustion
                        largely controlled by $HO_2$ and $H_2O_2$ pathways. Due
                        to higher pressures (10-30atm) and lower flame
                        temperatures (<1800K) as well as high collision
                        efficiency diluents (e.g., $CO_2, H_2O$). 
                    \item $HO_2$ and $H_2O_2$ pathways less well characterized
                        than branching reactions that dominate traditional
                        validation targets
                    \item Notes Hong model improves homogeneous target
                        performance considerably, but not high pressure /
                        dilute flame speeds
                    \item Sensitivity analysis suggested imprecise
                        experimental boundary and initial conditions
                        significant
                    \item Tested new model against all targets they used for 2004 Li mechanism + new targets
                    \item Note that Wang transport is better and should be
                        used, but that they could get just as good agreement
                        with chemkin compatible LJ transport
                    \item $H + O_2 (+M) = HO_2 (+M)$ competes with branching
                        reaction for H atoms, governing overall branching ratio.
                        Very well studied. Changed TBE for $H_2O$ and
                        broadening factor;  high pressure premixed flames
                        highly sensitive to TBE for $H_2O$ for this rxn. Also updated falloff behavior.
                    \item $H + O_2 = OH + O$. Branching reaction. Used new expression from Hong instead Hessler expression used in 2004.
                    \item $H+HO_2 = P$. Consumption of $HO_2$ and $H$ important
                        for higher pressures. Use theoretical result for 1
                        pathway then decreased it by 25\%. Other pathways
                        tweaked to get branching ratio reasonable. 
                    \item $OH + HO_2 = O_2 + H_2O$. Lean flames very sensitive
                        to this chain termination rxn. Used old rate constants,
                        because literature is uninformative about the correct
                        value, although non-Arrhenius behavior has been
                        identified between 900--1300K.
                    \item $H_2O (+M) = H + OH (+M)$. Large uncertainties.
                        Previously multiplied by 2. Needs to be bigger still;
                        tweaked to match experiments. 
                    \item Experimental data: shock tube speciation, ignition delay, laminar flame speeds, speciation in burner stabilized flames
                    \item Note high pressure lean flames highly sensitive to
                        neglected rxn X6 ($O + OH + M = HO_2 + M$). However,
                        present validation targets not sensitive\ldots and lack
                        of good estimates. (see Burke C\&F paper)
                \end{enumerate}

                \item[2013]$^S$
                  \citet{Keromnes_Many_Curran13}. Update to mechanism
                  of \cite{ConaireCSPW04}. Add RCM data.
                  \begin{enumerate}
                  \item \citet{MittalSY06} considered experimental
                    benchmark
                    \item From intro... reactivity of hydrogen
                      controlled by cometition between chain branch
                      rxn $H + O_2 = O + OH$ and pressure-dependent
                      chain-propagation reaction $H + O_2 + M = HO_2 +M$.
                    \item From intro... At high pressure, dominant
                      chain branching rxn is $H_2O_2 +M = OH + OH +
                      M$.
                    \item From intro... Prediction of ignition delay times flows
                      from $H_2 + HO_2 + H_2O_ 2+ H$
                    \item High pressure /low temperature in RCM
                      experiments high sensitivity to $H_2 + HO_2 =
                      H_2O_2 + H$
                      \item Pressure dependent rxns... R9 ($H + O_2 +
                        M  = HO_2 + M$) controls reactivity at `low
                        (<10atm)' pressure. At higher pressures, R15
                        ($H_2O_2 + M = OH +OH + M$) controls
                        reactivity. Ignition delay times highly
                        sensitive to this chain branching at
                        low-intermediate temperatures and high
                        pressures. 
                      \item $H+O_2 +M = HO_2 +M$ pressure dependent;
                        ref high-pressure flow cell of
                        \cite{FernandesLTU08}
                        \item Lean flame speeds sensitive to $H_2 + OH
                          = H + H_2O$.
                        \item $HO_2 + OH = H_2O + O_2$. Important
                          chain termination for fuel lean
                          flames. Well studied, but not well
                          known. non-Arrhenius behaviour c. 1200k.
                        \item $HO_2 + HO_2 = H_2O_2 + O_2$ noted as
                          inhibiting reactivity under low T, high P
                          conditions. 
                  \end{enumerate}
                  \item[2011] \citet{HongDH11} Mechanism update using
                    shock tube/laser absorption measurements
                    \begin{enumerate}
                    \item Focus on rate constants for: 
                      \begin{itemize}
                      \item $H + O_2 = OH + O$
                      \item $H_2O_2 +M = 2OH +M$
                      \item $OH + H_2O_2 = HO_2 + H_2O$
                      \item $O_2 + H_2O = OH + HO_2$
                      \end{itemize}
                    \item Note that $O + OH + M = HO_2 + M$ included
                      rarely (exception Williams mechanisms). Notes
                      Burke demonstrates this may impact lean
                      high-pressure flames, but choose to leave it out
                      because it is poorly known. 
                      \item Starting mechanism $H_2/O_2$ submechanism
                        from GRImech.
                    \end{enumerate}
\item[2014]$^S$ \citet{GoswamiBKG14} 
  \begin{enumerate}
  \item New experiments at 1--10atm using `heat flux method'; only
    $H_2-CO$ blends
  \item Update to the Konnov mechanism
  \end{enumerate}
            \item[2015]$^S$  \citet{LiYWL15}; reconsidered syngas mechanism taking H2 sub-mechanism from previous You et al. paper, CO submodel from Davis mechanism. 
                \begin{enumerate}
                    \item Tweaked rate parameters for $CO$ sub-mechanism for $CO + HO_2 = CO_2 + OH$ and $CO+OH = CO_2 + H$. 
                    \item Applied data collaboration method and found some of the experiments (at rich conditions) from Burke were inconsistent with the rest the model
                    \item Redid those experiments and found a consistent set
                \end{enumerate}

        \end{description}



\end{document}

